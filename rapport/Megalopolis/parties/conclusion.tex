\section{Conclusion}

A travers ce projet nous avons implémenté le jeu Megalopolis. Ce projet nous a permis de mieux prendre en main le langage TypeScript et surtout nous a permis d'appliquer les connaissances acquises en cours de programmation fonctionnelle.

\subsection{Ressenti sur le projet}

Nous avons trouvé ce projet encore plus intéressant et plus riche que celui du semestre 5. Se forcer à utiliser ces techniques de programmation particulières nous a montré que dans certains cas les fonctions pouvaient être codées beaucoup plus naturellement et facilement. Notamment l'utilisation des \texttt{map} et des \texttt{reduce} s'est révélée naturelle.

De plus ce sujet de TS étant très libre, nous avons pu discuter sur la façon de faire, échanger nos points de vu et finir par poser notre propre implémentation en connaissance de cause. Ce qui nous a vraiment plu car on avait une compréhension fine du projet.

\subsection{Retour sur les attendus}

A travers ce projet, nous avons appliqué ce que nous avons appris en cours de programmation fonctionnelle. Voici notamment les principaux points~: l'utilisation des typages, l'utilisation de la librairie \texttt{Immutable.js}, la pris en main de \texttt{jest}, de \texttt{ESLint} et de \texttt{Parcel} et comme dit précédemment la prise en main des fonctions récursives communes tel que \texttt{map} et \texttt{reduce}. Finalement c'est avec ces attendus que nous avons réussi à implémenter le jeu Megalopolis.
