\section{Répartition du travail}

Pour trois d'entre-nous c'était la première fois que nous faisions un projet par groupe de quatre, cependant la division du travail s'est fait naturellement. Nous avons commencé par travailler en peer-programming afin d'être sur de tous partir sur les mêmes bases puis nous nous sommes chacun répartis les tâches et si nécessaire se remettre en peer-programming pour résoudre des problèmes plus complexe.

%pose syndical, goûter.
\subsection{Peer-programming}
Le choix du Peer-programming a été fait afin de confronter nos idées et ainsi trouver la meilleure implémentation possible. Le projet nous laissant libre sur certaines implémentations, il fallait que chaque membre du groupe puisse participer leurs conceptions. Par ailleurs d'après le cours PG106, cette stratégie permet de limiter les erreurs dans le code, étant donné qu'il y a au moins 2 relecteurs pour un même code. Cela nous a permis d'appréhender une nouvelle technique de programmation.

